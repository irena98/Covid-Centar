\documentclass[titlepage]{article}
\usepackage{xcolor}
\usepackage[english, croatian]{babel}

\usepackage{verbatim}
\usepackage{graphicx}
\usepackage{float}
\graphicspath{ {./Dijagrami_slike/} }

\title{INFORMACIONI SISTEMI\\Kovid sistem}
\author{
Bogdan Bojović 1019/2021\\
Kosta Grujčić 1012/2021\\
Miodrag Radojević 1012/2020\\
Luka Đorović 1029/2021\\
Irena Vasiljević 1018/2021
}

\date{Beograd, 2021.}

\begin{document}

\maketitle
\tableofcontents

\newpage

\section{Uvod}

\section{Analiza sistema}

Informacioni sistem Kovid centar svojim funkcionalnostima obezbe\dj{}uje sigurno i brzo sprovo\dj{}enje procedura neophodnih za za\v{s}titu od virusa COVID-19. Namenjen je pre svega osobama koje nameravaju da se testiraju i/ili vakcini\v{s}u kao i nadle\v{z}nim licima za kori\v{s}\'{c}enje relevantnih statistika.\newline
\indent Rezultati testova, informacije o testiranim licima, informacije o vakcinisanim licima i drugo, predstavljaju podatke koje na\v{s} informacioni sistem koristi pri a\v{z}uriranju korisnika o narednim koracima za\v{s}tite od virusa kao i pri dodeljivanju kovid propusnica. Pored toga, na zahtev mo\v{z}emo podatke dostaviti nadle\v{z}nim licima iz ministarstva zdravlja.\newline
\indent Na Slici \ref{slk:slucajevi} nalazi se dijagram koji prikazuje učesnike sistema i njihove poslove. Slika \ref{slk:kontekst} prikazuje dijagram interakcije sistema sa svetom, a na Slici \ref{slk:dtp} se nalazi dijagram koji opisuje glavne procese i tok podataka u sistemu.

\begin{figure}[H]
\centering
\includegraphics[scale=0.5]{Dijagram_slucajeva_upotrebe}
\caption{Dijagram slučajeva upotrebe}
\label{slk:slucajevi}
\end{figure}

\begin{figure}[H]
\centering
\includegraphics[scale=0.5]{Dijagram_konteksta}
\caption{Dijagram konteksta}
\label{slk:kontekst}
\end{figure}

\begin{figure}[H]
\centering
\includegraphics[scale=0.45]{DTP_dijagram}
\caption{Dijagram toka podataka nivoa 0}
\label{slk:dtp}
\end{figure}


\section{Procesi i slučajevi upotrebe}

%Menjajte strukturu i nazive kako vam se uklapa u slučaj :)

\subsection{Prva doza vakcine - aktivnosti}
Ovo poglavlje sačinjeno je od formalno predstavljenih slučajeva upotrebe počev od podnošenja prijave do evidentiranja uspešno primljene prve doze vakcine.

\subsubsection{Slučaj upotrebe: Online registracija}
\begin{itemize}
    \item \textbf{Kratak opis:} Neregistrovana osoba vrši registraciju popunjavanjem online forme, traženim podacima. Sistem potvrđuje validnost podataka i vraća potvrdu o uspešnosti registracije.
    \item \textbf{Učesnici:}
        \begin{itemize}
            \item Neregistrovana osoba - želi da se registruje u sistem, u cilju podnošenja zahteva za vakcinaciju.
        \end{itemize}
    \item \textbf{Preduslovi:} Sistem je aktivan. Neregistrovana osoba ima pristup internetu.
    \item \textbf{Postuslovi:} Osoba je registrovana i ima pristup sistemu. Njeni podaci su sačuvani u bazi. Dobila je korisničko ime i lozinku uz pomoć kojih se loguje na sistem.
    \item \textbf{Osnovni tok:}
        \begin{enumerate}
            \item Osoba otvara web-stranicu za registraciju.
	    \item Sistem prikazuje formular za registraciju.
	    \item Osoba unosi tražene podatke.
	    \item Osoba potvrđuje unos.
	    \item Sistem vrši validaciju podataka.
	    \item Sistem čuva podatke.
	    \item Sistem pravi privremeni nalog.
	    \item Sistem šalje osobi email u kojem traži potvrdu registracije.
	    \item Sistem obaveštava osobu da je email poslat.
	    \item Osoba proverava poštu i potvrđuje registraciju.
	    \item Sistem privremeni nalog trajno aktivira.
            \item Sistem obaveštava osobu da je uspešno registrovana.
	\end{enumerate}
     
    %Ako postoje, podtokove nazivati sa P1, P2, ...   
    %\item \textbf{Podtokovi:}    
    
    %Alternativne tokove nazivati sa A1, A2, ...
    \item \textbf{Alternativni tokovi:}
        \begin{itemize}
            \item[A1.] \textbf{Neuspešna validacija podataka.} Ukoliko u koraku 5 osnovnog toka sistem naiđe na neispravne podatke, obaveštava korisnika i zahteva ponovni unos onih polja gde su nevalidni podaci. Pri čemu označava korisniku polja sa nevalidnim podacima. Kada osoba unese sve podatke ispravno, proces se nastavlja u koraku 4 osnovnog toka.
            \item[A2.] \textbf{Nije stigao email za potvrdu registracije.} Ukoliko osoba nije dobila email za potvrdu, klikom na dugme zahteva ponovno slanje emaila. Proces se nastavlja u koraku 8.
	    \item[A3.] \textbf{Link za registraciju je istekao.} Ukoliko osoba nije potvrdila registraciju u predviđenom periodu, sistem briše privremeni nalog. Proces se završava.
        \end{itemize}
    
    %Ako postoje, specijalne zahteve navesti ovde
    \item \textbf{Specijalni zahtevi:}
		\begin{itemize}
			\item Potrebno je da je osoba koja se registruje punoletna.
		\end{itemize}
  
    \item \textbf{Dodatne informacije:}
        \begin{itemize}
            \item  Podaci potrebni za prijavu su:
                \begin{itemize}
                    \item ime
                    \item prezime
                    \item JMBG
                    \item broj zdravstvene knjižice
                    \item broj telefona
                    \item email adresa
		    \item korisničko ime
		    \item lozinka
                \end{itemize}
        \end{itemize}

\end{itemize}


\subsubsection{Slučaj upotrebe: Podnošenje prijave online}
\begin{itemize}
    \item \textbf{Kratak opis:} Neprijavljena osoba vrši prijavu popunjavanjem online forme, traženim podacima. Sistem potvrđuje validnost podataka i vraća poruku o uspešnosti prijave.
    \item \textbf{Učesnici:}
        \begin{itemize}
            \item Neprijavljena osoba - želi da se prijavi za prvu dozu vakcine.
        \end{itemize}
    \item \textbf{Preduslovi:} Sistem je aktivan. Neprijavljena osoba je registorvana kao korisnik sistema i ima pristup internetu.
    \item \textbf{Postuslovi:} Korisnik je dobio potvrdu o zakazanom terminu i mestu prve vakcinacije.
    \item \textbf{Osnovni tok:}
        \begin{enumerate}
            \item Osoba otvara web-stranicu za logovanje.
	    \item Osoba unosi i potvrđuje korisničko ime i lozinku.
            \item Sistem prikazuje formular za prijavu.
            \item Osoba bira ponuđene ustanove i proizvođače vakcina.
            \item Osoba potvrđuje unos.
            \item Sistem čuva unete podatke.
	    \item Sistem daje mogućnost izbora raspoloživih termina za zadate ustanove i vakcine.
	    \item Osoba bira ponuđene termine i potvrđuje. 
            \item Sistem zakazuje termin vakcinacije.
	    \item Sistem korisniku šalje email sa potvrdom o zakazanom terminu.
            \item Sistem obaveštava korisnika da je operacija uspešno izvršena i da je potvrda poslata.
	\end{enumerate}
     
    %Ako postoje, podtokove nazivati sa P1, P2, ...   
    %\item \textbf{Podtokovi:}    
    
    %Alternativne tokove nazivati sa A1, A2, ...
    \item \textbf{Alternativni tokovi:}
        \begin{itemize}
            \item[A1.] \textbf{Neuspešno logovanje.} Ukoliko u koraku 2 osnovnog toka sistem naiđe na neispravne podatke, obaveštava korisnika i zahteva ponovni unos korisničkog imena i lozinke. Proces se nastavlja u koraku 2 osnovnog toka.
        \end{itemize}
    
    %Ako postoje, specijalne zahteve navesti ovde
    \item \textbf{Specijalni zahtevi:}
		\begin{itemize}
			\item Potrebno je da je zdravstvena knjižica, osobe koja zakazuje vakcinaciju, u trenutku zakazivanja važeća.
		\end{itemize}
  
\end{itemize}

\begin{figure}[H]
\centering
\includegraphics[scale=0.7]{Podnosenje_prijave_online}
\caption{Dijagram aktivnosti - Pondnošenje prijave online}
\end{figure}

\subsubsection{Slučaj upotrebe: Podnošenje prijave uživo}
\begin{itemize}
    \item \textbf{Kratak opis:} Neprijavljena osoba vrši prijavu popunjavanjem papirnog formulara, traženim podacima. Sistem potvrđuje validnost podataka i vraća poruku o uspešnosti prijave.
    \item \textbf{Učesnici:}
        \begin{itemize}
            \item Neprijavljena osoba - želi da se prijavi za prvu dozu vakcine.
	    \item Medicinski radnik - pruža pomoć osobi u podnošenju prijave.
        \end{itemize}
    \item \textbf{Preduslovi:} Sistem je aktivan.
    \item \textbf{Postuslovi:} Osoba je dobila potvrdu o zakazanom terminu i mestu prve vakcinacije.
    \item \textbf{Osnovni tok:}
        \begin{enumerate}
            \item Osoba dolazi na informacioni pult i izjavljuje da želi da se vakciniše.
	    \item Medicinski radnik na pultu odlazi na deo sistema koji je predviđen za registraciju novih korisnika.
	    \item Osoba daje potrebne informacije za registraciju, zdravstvenu knjižicu i neki lični dokument za evidenciju medicinskom radniku.
	    \item Medicinski radnik unosi podatke u sistem.
	    \item Medicinski radnik potvrđuje unos.
	    \item Sistem vrši validaciju podataka.
            \item Sistem čuva podatke.
            \item Sistem prikazuje informacije o nalogu koji kreira i traži potvrdu registracije.
	    \item Medicinski radnik nakon provere podataka potvrđuje registraciju osobe.
            \item Sistem obaveštava da je osoba uspešno registrovana.
	    \item Medicinski radnik od osobe traži da odabere ustanove i proizvođače vakcina.
	    \item Osoba obaveštava medicinskog radnika o odgovarajućim ustanovama i proizvođačima vakcine.
	    \item Medicinski radnik unosi podatke u sistem i potvrđuje.
	    \item Sistem dajte pregled slobodnih termina.
            \item Medicinski radnik obaveštava osobu o slobodnim terminima.
	    \item Osoba bira termin.
	    \item Medicinski radnik bira odgovarajući termin u sistemu i potvrđuje.
            \item Sistem zakazuje termin vakcinacije.
	    \item Sistem osobi šalje email sa potvrdom o zakazanom terminu.
            \item Sistem obaveštava medicinskog radnika da je operacija uspešno izvršena i da je potvrda poslata.
	    \item Medicinski radnik obaveštava osobu da je operacija uspešno izvršena i da je potvrda poslata na njen email.
	\end{enumerate}
     
    %Ako postoje, podtokove nazivati sa P1, P2, ...   
    %\item \textbf{Podtokovi:}    
    
    %Alternativne tokove nazivati sa A1, A2, ...
    \item \textbf{Alternativni tokovi:}
        \begin{itemize}
            \item[A1.] \textbf{Neuspešna validacija podataka.} Ukoliko u koraku 6 osnovnog toka sistem naiđe na neispravne podatke, obaveštava medicinskog radnika i zahteva ponovni unos onih polja gde su nevalidni podaci. Pri čemu označava polja sa nevalidnim podacima. Kada medicinski radnik unese sve podatke ispravno, proces se nastavlja u koraku 5 osnovnog toka.
	     \item[A2.] \textbf{Nema slobodnih termina.} Ukoliko u koraku 14 osnovnog toka medicinski radnik dobije obaveštenje od sistema da nema slobodnih termina za zadate ustanove i vakcine, on informaciju prenosi osobi i proces se nastavlja u koraku 11 osnovnog toka.
        \end{itemize}
    
    %Ako postoje, specijalne zahteve navesti ovde
    \item \textbf{Specijalni zahtevi:}
		\begin{itemize}
			\item Potrebno je da je zdravstvena knjižica, osobe koja zakazuje vakcinaciju, u trenutku zakazivanja važeća.
		\end{itemize}


     \item \textbf{Dodatne informacije:}
        \begin{itemize}
	    \item Dokument za evidenciju osobe čija se registracija vrši može biti pasoš ili lična karta.
            \item  Podaci potrebni za prijavu su:
                \begin{itemize}
                    \item ime
                    \item prezime
                    \item JMBG
                    \item broj zdravstvene knjižice
                    \item broj telefona
                    \item email adresa
		    \item korisničko ime
		    \item lozinka
                \end{itemize}
        \end{itemize}
  
\end{itemize}


\subsubsection{Slučaj upotrebe: Otkazivanje termina vakcinacije - online prijavljena osoba}
\begin{itemize}
    \item \textbf{Kratak opis:} Online prijavljena osoba za primanje prve doze vakcine, usled izvesnih okolnosti otkazuje zauzeti termin za primanje prve doze vakcine. Sistem potvrđuje odjavu i oslobađa termin.
    \item \textbf{Učesnici:}
        \begin{itemize}
            \item Prijavljena osoba - želi da se otkaže termin primanja prve doze vakcine.
        \end{itemize}
    \item \textbf{Preduslovi:} Sistem je aktivan. Osoba je prethodno zakazala primanje prve doze vakcine.
    \item \textbf{Postuslovi:} Osoba je dobila potvrdu o otkazanom terminu i ima mogućnost da se ponovo prijavi za primanje prve doze vakcine.
    \item \textbf{Osnovni tok:}
        \begin{enumerate}
            \item Osoba otvara web-stranicu za logovanje.
	    \item Osoba unosi i potvrđuje korisničko ime i lozinku.
	    \item Sistem prikazuje prethodno podnetu prijavu i informacije o njoj.
	    \item Osoba bira opciju za otkazivanje prijave i potvrđuje je.
	    \item Sistem otvara formu za unos razloga otkazivanja prijave.
	    \item Sistem zahteva od osobe da popuni formu.
	    \item Osoba unosi razlog otkazivanja prijave.
            \item Osoba potvrđuje unos razloga.
	    \item Sistem evidentira odjavu.
	    \item Sistem oslobađa termin za dato mesto, vakcinu i vreme.
	    \item Sistem osobi šalje email sa potvrdom o uspešnoj odjavi termina.
	    \item Sistem obaveštava osobu o uspešnoj odjavi termina i da je potvrda o odjavi poslata.
	\end{enumerate}
     
    %Ako postoje, podtokove nazivati sa P1, P2, ...   
    %\item \textbf{Podtokovi:}    
    
    %Ako postoje, specijalne zahteve navesti ovde
    \item \textbf{Specijalni zahtevi:}
		\begin{itemize}
			\item Potrebno je da osoba koja se odjavljuje ima validan razlog za otkazivanje prve primanja prve doze vakcine.
		\end{itemize}
\end{itemize}


\subsubsection{Slučaj upotrebe: Otkazivanje termina vakcinacije - uživo prijavljena osoba}
\begin{itemize}
    \item \textbf{Kratak opis:} Uživo prijavljena osoba za primanje prve doze vakcine, usled izvesnih okolnosti otkazuje zauzeti termin za primanje prve doze vakcine. Sistem potvrđuje odjavu i oslobađa termin.
    \item \textbf{Učesnici:}
        \begin{itemize}
            \item Prijavljena osoba - želi da se otkaže termin primanja prve doze vakcine.
	    \item Medicinski radnik - pruža pomoć osobi u podnošenju odjave.
        \end{itemize}
    \item \textbf{Preduslovi:} Sistem je aktivan. Osoba je prethodno zakazala primanje prve doze vakcine.
    \item \textbf{Postuslovi:} Osoba je dobila potvrdu o otkazanom terminu i ima mogućnost da se ponovo prijavi za primanje prve doze vakcine.
    \item \textbf{Osnovni tok:}
        \begin{enumerate}
            \item Osoba dolazi na informacioni pult ili telefonskim pozivom izjavljuje da želi da odjavi zauzeti termin.
	    \item Medicinski radnik na pultu odlazi na deo sistema koji je predviđen za odjavu prijavljenih osoba.
	    \item Sistem zahteva podatke osobe koja želi da odjavi termin.
	    \item Medicinski radnik zahteva podatke osobe koja se odjavljuje.
	    \item Osoba obaveštava medicinskog radnika o podacima.
	    \item Medicinski radnik unosi podatke u sistem i potvrđuje.
            \item Sistem za date podatke pronalazi registrovani nalog.
	    \item Sistem prikazuje nalog i prethodno zakazani termin.
	    \item Sistem zahteva unošenje razloga i potvrdu odjave.
	    \item Medicinski radnik proverava nalog osobe i prethodno zakazani termin.
	    \item Medicinski radnik zahteva razlog odjave termina od osobe.
	    \item Osoba obaveštava medicinskog radnika o razlogu odjave termina.
	    \item Medicinski radnik unosi razlog odjave termina i potvrđuje odjavu termina.
	    \item Sistem evidentira odjavu.
	    \item Sistem oslobađa termin za dato mesto, vakcinu i vreme.
	    \item Sistem obaveštava medicinskog radnika o uspešnoj odjavi termina.
	    \item Medicinski radnik obaveštava osobu o uspešnoj odjavi termina.
	\end{enumerate}
     
    %Ako postoje, podtokove nazivati sa P1, P2, ...   
    %\item \textbf{Podtokovi:}    
    
    %Alternativne tokove nazivati sa A1, A2, ...
    \item \textbf{Alternativni tokovi:}
        \begin{itemize}
            \item[A1.] \textbf{Nepostojeći podaci u bazi.} Ukoliko u koraku 7 osnovnog toka sistem za unete podatke ne pronađe nalog u bazi, obaveštava medicinskog radnika i zahteva ponovni unos podataka. Medicinski radnik obaveštava osobu o neispravnosti podataka i zahteva da se podaci ponovo daju. Proces se nastavlja u koraku 4 osnovnog toka.
        \end{itemize}
    
    %Ako postoje, specijalne zahteve navesti ovde
    \item \textbf{Specijalni zahtevi:}
		\begin{itemize}
			\item Potrebno je da osoba koja se odjavljuje ima validan razlog za otkazivanje prve primanja prve doze vakcine.
		\end{itemize}
\end{itemize}

\subsection{Naredne doze vakcine - aktivnosti}
\subsubsection{Slučaj upotrebe: Prijava i obaveštavanje o narednoj dozi vakcine}
\begin{itemize}
    \item \textbf{Kratak opis:} Sistem na osnovu podataka o primanju prethodne doze vakcine osobu prijavljuje za dobijanje naredne doze vakcine i obaveštava je o terminu.
    \item \textbf{Učesnici:} 
        \begin{itemize}
            \item Osoba - osoba koja treba da primi narednu dozu vakcine.
        \end{itemize}
    \item \textbf{Preduslovi:} Sistem je aktivan, podaci o vakcinaciji osobe su ažurni.
    \item \textbf{Postuslovi:} Osoba je prijavljena za dobijanje naredne doze vakcine i o tome biva obaveštena. Sistem je ažuriran.
    \item \textbf{Osnovni tok:}
    \begin{enumerate}
        \item Sistem pristupa podacima o vakcinisanima. Sistem ovu operaciju izvodi automatski, svakog dana u podne.
        \item Sistem predla\v{z}e vreme i mesto vakcinacije drugom dozom vakcine koja je već primljena prvi put.
        \item Osoba ne prihvata vreme i mesto vakcinacije.
        \item Sistem šalje formular osobi kako bi odabrala vreme i mesto vakcinacije
        \item Osoba upisuje vreme i mesto vakcinacije.
        \item Sistem validira unete podatke.
        \item Sistem obave\v{s}tava osobu da je termin zakazan.
        \item Sistem a\v{z}urira podatke o zakazanim terminima (paralelno sa korakom 7).
    \end{enumerate}
    \item \textbf{Alternativni tokovi:}
    \begin{itemize}
        \item[A1.] \textbf{Osoba prihvata predlo\v{z}eni termin.} Ukoliko osoba u koraku 3 prihvati vreme i mesto koje je sistem po automatizmu poslao, šalje potvrdu sistemu i proces se nastavlja u tački 7 osnovnog toka. 
        \item[A2.] \textbf{Sistem ne prihvata zadati termin.} Ako sistem ne uspe da izvrši validaciju zadatog termina u tački 6, \v{s}alje isti formular osobi. 
        Proces se nastavlja u tački 4 osnovnog toka.
        
    \end{itemize}
    \item \textbf{Dodatne informacije:}
    \begin{itemize}
        \item Sistem treba ponuditi mogućnosti osobi koja želi da popuni formular, u vidu spiska lokacija Kovid-ambulanti i spiska termina (datum i vreme - na svakih pola sata u okviru radnog vremena).
    \end{itemize}
\end{itemize}


\begin{figure}[H]
\centering
\includegraphics[scale=0.8]{Vakcinacija_drugom_dozom}
\caption{Dijagram aktivnosti - Prijava i obaveštavanje o narednoj dozi vakcine}
\end{figure}

\subsection{Vakcinacija}
\subsubsection{Slučaj upotrebe: Vakcinacija i evidentiranje}
\begin{itemize}
	\item \textbf{Kratak opis}: Osoba je prijavljena za vakcinaciju bilo kojom dozom. Nakon završene vakcinacije se sistem ažurira.
	\item \textbf{Učesnici}:
	\begin{itemize}
		\item Osoba - osoba koja se vaksiniše.
		\item Medicinski radnik - službeno lice koje vakciniše i upisuje potrebne informacije.
	\end{itemize}
	\item \textbf{Preduslovi}: Osoba ja prijavljena i ima zakazan termin. Sistem je aktivan.
	\item \textbf{Postuslovi}: Osoba je vaksinisana. Sistem je ažuriran.
	\item \textbf{Osnovni tok}:
	\begin{enumerate}
		\item Osoba medicinskom radniku daje svoju zdravstvenu knjižicu i ostale informacije potrebne za identifikaciju.
		\item Proverava se da li je osoba prijavljena za vakcinaciju.
		\item Medicinski radnik vaksiniše osobu.
		\item Medicinski radnik upisuje potrebne informacije u sistem.
	\end{enumerate}
	\item \textbf{Alternativni tokovi}:
	\begin{enumerate}
		\item[A1.] \textbf{Osoba nije prijavljena}. Ukoliko se u koraku 2 osnovnog toka ustanovi da osoba nije prijavljena, onda ona ne može biti vakcinisana.
	\end{enumerate}
\end{itemize}

\begin{center}
	\begin{figure}[H]
		\includegraphics[width=0.6\textwidth]{Vakcinacija}
		\caption{Dijagram aktivnosti - Vakcinacija i evidentiranje}
	\end{figure}
\end{center}

\subsection{Informisanje}
\subsubsection{Slučaj upotrebe: Sprovođenje statistike nadle\v{z}nom licu}
\begin{itemize}
\item \textbf{Kratak opis:}  Nadle\v{z}no lice iz ministarstva zdravlja popunjava formular na osnovu kojeg mu se nakon autentikacije i konsultovanja baze podataka u malom broju koraka dostavljaju tra\v{z}eni podaci.
\item \textbf{Učesnici:}
\begin{itemize}
    \item Nadle\v{z}no lice iz ministarstva iz zdravlja.
\end{itemize}
 \item \textbf{Preduslovi:} Sistem je aktivan. Potra\v{z}ilac statistike iz ministarstva zdravlja ima pristup internetu.
 \item \textbf{Postuslovi:} Ministarstvu su na raspolaganje dostavljeni zatra\v{z}eni podaci.
 \item \textbf{Osnovni tok:}
 \begin{enumerate}
    \item Lice iz ministarstva otvara stranicu za autentikaciju.
    \item Sistem prikazuje formular za autentikaciju.
    \item Lice iz ministarstva popunjava polja za: ime, prezime, JMBG i jedinstveni identifikacioni broj koji je specijalno dodeljen za ovaj vid transakcije.
    \item Lice potvrđuje unos.
    \item Sistem proverava informacije koje su mu dostavljene u popunjenom formularu za autentikaciju.
    \item Nakon provere identiteta, sistem obave\v{s}tava lice iz ministarstva o uspe\v{s}nosti autentikacije.
    \item Sistem prikazuje stranicu sa ponu\dj{}enim opcijama za dohvatanje statistike.
    \item Lice iz ministarstva bira neke od opcija:
    \begin{itemize}
                    \item Dohvatanje statistike o broju vakcinisanih Pfizer-BioNTech vakcinom u odre\dj{}enom vremenskom periodu.
                    \item Dohvatanje statistike o broju vakcinisanih Sputnik V vakcinom u odre\dj{}enom vremenskom periodu.
                    \item Dohvatanje statistike o broju vakcinisanih Sinopharm vakcinom u odre\dj{}enom vremenskom periodu.
                    \item Dohvatanje statistike o broju vakcinisanih Oxford/AstraZeneca vakcinom u odre\dj{}enom vremenskom periodu.
                    \item Dohvatanje statistike o broju testiranih na COVID-19 u odre\dj{}enom vremenskom periodu.
                    \item Dohvatanje statistike o broju osoba sa pozitivnim testom na \newline COVID-19 u odre\dj{}enom vremenskom periodu.
                    \item Dohvatanje informacija (vreme, identitet lica iz ministarstva) o prethodnim potra\v{z}ivanjima iz ministarstva zdravlja.
                \end{itemize}
    \item Sistem proverava da li je uspe\v{s}no izabrana neka od opcija.
    \item Sistem na osnovu izbora dostavlja statistiku licu iz ministartstva zdravlja.
    \item Informacije o vremenu i identitetu potra\v{z}ioca \v{c}uvaju se u sistemu.
 \end{enumerate}
 \item \textbf{Alternativni tokovi:}
 \begin{itemize}
            \item[A1.] \textbf{Neuspe\v{s}na dodela dozvole za podno\v{s}eje zahteva.} Ukoliko u koraku 5 osnovnog toka sistem naiđe na neispravne podatke, obaveštava korisnika i zahteva ponovni unos podataka. Proces se nastavlja u koraku 3 osnovnog toka.
            \item[A2.] \textbf{Neispravno podno\v{s}enje zahteva.} Ukoliko u koraku 9 osnovnog toka sistem utvrdi da lice iz ministarstva nije odabralo nijednu od ponu\dj{}enih opcija, lice o tome biva obave\v{s}teno. Proces se nastavlja u koraku 8 osnovnog toka. 
        \end{itemize}
 \item \textbf{Dodatne informacije:}
            \begin{itemize}
                \item Jedinstveni identifikacioni broj koji se koristi za autentikaciju u\v{c}esnika i dodelu dozvole za podno\v{s}enje zahteva za statistiku dodeljuje se u dogovoru sa ministarstvom zdravlja posebno svim licima koja \'{c}e biti  zadu\v{z}ena za dohvatatanje statistike.
            \end{itemize}

\end{itemize}

\begin{figure}[H]
\includegraphics[scale=0.53]{Sprovodjenje_statistike}
\caption{Dijagram aktivnosti - Sprovo\dj{}enje statistike nadle\v{z}nom licu}
\end{figure}

\subsubsection{Slučaj upotrebe: Kol centar}
\begin{itemize}
\item \textbf{Kratak opis:} Osoba poziva kol centar Kovid sistema u cilju informisanja. Dodeljenom opereteru može postavljati pitanja koja se tiču terapije, vakcinacije, simptoma, tegoba (naročito posle vakcinacije). Saopštene informacije se čuvaju u sistemu za potrebe dalje analize i izrade statistike u okviru sistema.
\item \textbf{Učesnici:}
\begin{itemize}
    \item Osoba koja poziva kol centar.
    \item Operater koji se dodeljuje osobi za vreme poziva.
\end{itemize}
 \item \textbf{Preduslovi:} Sistem je aktivan. Osoba poseduje mobilni ili fiksni telefon. Postoje aktivni operateri.
 \item \textbf{Postuslovi:} Osoba je dobila odgovore na sva pitanja koja je postavila operateru. Sistem ažurira podatke o tekućem razgovoru.
 \item \textbf{Osnovni tok:}
 \begin{enumerate}
    \item Osoba poziva broj telefona kol centra.
    \item Neki od trenutno slobodnih operatera se dodeljuje osobi za vreme poziva.
    \item Osoba i operater otpočinju razgovor u kom operater odgovara na postavljena pitanja.
    \item Razgovor se završava.
    \item Sistem ažurira podatke o simptomima i tegobama.
 \end{enumerate}
 \item \textbf{Alternativni tokovi:}
 \begin{itemize}
            \item[A1.] \textbf{Prekid veze} Ukoliko nakon koraka 2 osnovnog toka sistema dođe do tehničkih smetnji ili obustave poziva, razgovor biva prekinut.
        \end{itemize}
 \item \textbf{Dodatne informacije:}
            \begin{itemize}
                \item Kol centar poseduje jedinstveni servisni broj otvoren od strane regulatornog tela.
                \item Svi pozivaoci bivaju preusmereni na određeni lokal, dok se operater dodeljuje prema listi čekanja.
            \end{itemize}

\end{itemize}

\begin{figure}[H]
\includegraphics[width=\textwidth,height=\textheight,keepaspectratio]{Kol_centar}
\caption{Dijagram aktivnosti - Kol centar}
\end{figure}

\subsection{Kovid propusnice}

U ovom poglavlju se bavimo formalizacijom rada sa kovid propusnicama. U nastavku su opisani slučajevi upotrebe izdavanja i validacije kovid propusnica.

\subsubsection{Slučaj upotrebe: Izdavanje kovid propusnice}

%Prvih 5 tacaka je obavezno, ostale nisu - navesti potrebne
%Ne bi bilo loše da, ako se dodaje neki dijagram, stoji opis uz njega i da se negde u tekstu onda reveriše na tu sliku

\begin{itemize}
    \item \textbf{Kratak opis:} Registrovana osoba bira opciju za podnošenje zahteva za kovid propusnicu. Sistem proverava podatke, izdaje propusnicu i vraća odgovarajuću poruku.
    \item \textbf{Učesnici:}
        \begin{itemize}
            \item Registrovana osoba - želi brzo da dobije kovid propusnicu uz minimalan broj koraka
        \end{itemize}
    \item \textbf{Preduslovi:} Sistem je aktivan. Osoba je registrovana i ima pristup internetu.
    \item \textbf{Postuslovi:} Sistem je izdao propusnicu registrovanoj osobi.
    \item \textbf{Osnovni tok:}
        \begin{enumerate}
            \item Osoba otvara stranicu za prijavu.
            \item Sistem prikazuje formular za prijavu.
            \item Osoba unosi odgovarajuće podatke.
            \item Osoba potvrđuje unos.
            \item Sistem vrši validaciju podataka.
            \item Osoba otvara stranicu za podnošenje zahteva za izdavanje propusnice.
            \item Sistem prikazuje tri moguće opcije:
                \begin{itemize}
                    \item Izdavanje propusnice na osnovu primljene vakcine.
                    \item Izdavanje propusnice na osnovu negativnog PCR ili antigenskog testa.
                    \item Izdavanje propusnice na osnovu preležanog virusa.
                \end{itemize}
            \item Osoba bira jednu od ponuđenih opcija.
            \item Sistem proverava zadovoljenost uslova.
            \item Sistem osobi šalje mejl sa kovid propusnicom.
            \item Sistem obaveštava osobu da je operaciju uspešno izvršena i da je propusnica poslata.
        \end{enumerate}
     
    %Ako postoje, podtokove nazivati sa P1, P2, ...   
    %\item \textbf{Podtokovi:}    
    
    %Alternativne tokove nazivati sa A1, A2, ...
    \item \textbf{Alternativni tokovi:}
        \begin{itemize}
            \item[A1.] \textbf{Neuspešno prijavljivanje.} Ukoliko u koraku 5 osnovnog toka sistem naiđe na neispravne podatke, obaveštava osobu i zahteva ponovni unos podataka. Proces se nastavlja u koraku 3 osnovnog toka.
            \item[A2.] \textbf{Uslovi za izdavanje nisu zadovoljeni.} Ukoliko u koraku 9 osnovnog toga sistem ustanovi da nisu zadovoljeni odgovarajući uslovi za izdavanje propusnice, obaveštava osobu. Proces se završava.
        \end{itemize}
    
    %Ako postoje, specijalne zahteve navesti ovde
    %\item \textbf{Specijalni zahtevi:}
        
    \item \textbf{Dodatne informacije:}
        \begin{itemize}
            \item Podaci koji su potrebni za prijavu na sistem su korisničko ime i lozinka.
            \item Uslovi za izdavanje propusnice su:
                \begin{itemize}
                    \item Druga doza vakcine je primljena pre manje od 7 meseci ili  je primljena treća doza vakcine.
                    \item Postojanje negativnog PCR testa koji nije stariji od 72 sata ili antigenskog testa koji nije stariji od 48 sati.
                    \item Virus je preležan pre manje od 7 meseci.
                \end{itemize}
            \item Kovid propusnica sadrži QR kod na osnovu kog se proverava validnost propusnice.
        \end{itemize}
\end{itemize}

\begin{figure}[H]
\centering
\includegraphics[scale=0.25]{Izdavanje_propusnice}
\caption{Dijagram aktivnosti - Izdavanje kovid propusnice}
\label{slk:izdavanje}
\end{figure}

\subsubsection{Slučaj upotrebe: Validacija kovid propusnica}

\begin{itemize}
    \item \textbf{Kratak opis:} Kontrolor očitava QR kod sa kovid propusnice. Sistem proverava podatke i prikazuje odgovarajuću poruku o validnosti propusnice.
    \item \textbf{Učesnici:}
        \begin{itemize}
            \item Kontrolor - želi brzo da proveri validnost kovid propusnice
        \end{itemize}
    \item \textbf{Preduslovi:} Sistem je aktivan. Kontrolor ima pristup internetu.
    \item \textbf{Postuslovi:} Sistem je obavestio kontrolora o validnosti kovid propusnice.
    \item \textbf{Osnovni tok:}
        \begin{enumerate}
            \item Kontrolor očitava QR kod sa kovid propusnice.
            \item Kontrolor otvara stranicu za validaciju.
            \item Sistem dobija zahtev za validaciju propusnice.
            \item Sistem proverava zadovoljenost uslova za važenje propusnice.
            \item Sistem prikazuje podatke o vlasniku propusnice i status propusnice:
                \begin{itemize}
                    \item Zeleno - ukoliko su uslovi za važenje propusnice zadovoljeni.
                    \item Crveno - ukoliko uslovi za važenje propusnice nisu zadovoljeni.
                \end{itemize}
        \end{enumerate}
    \item \textbf{Specijalni zahtevi:}
        \begin{itemize}
            \item Kontrolor poseduje uređaj kojim se može skenirati QR kod.
        \end{itemize}
    \item \textbf{Dodatne informacije:}
        \begin{itemize}
            \item Uslovi za važenje propusnice su:
                \begin{itemize}
                    \item Druga doza vakcine je primljena pre manje od 7 meseci ili  je primljena treća doza vakcine.
                    \item Postojanje negativnog PCR testa koji nije stariji od 72 sata ili antigenskog testa koji nije stariji od 48 sati.
                    \item Virus je preležan pre manje od 7 meseci.
                \end{itemize}
        \end{itemize}
\end{itemize}

\begin{figure}[H]
\centering
\includegraphics[scale=0.5]{Validacija_propusnice}
\caption{Dijagram sekvence - Validacija kovid propusnice}
\label{slk:validacija}
\end{figure}

\subsection{Testiranje}

\subsubsection{Slučaj upotrebe: Unos informacija o urađenom testu}

\end{document}
