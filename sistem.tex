\documentclass[titlepage]{article}
\usepackage{xcolor}
\usepackage[serbian]{babel}
\usepackage[T2A]{fontenc}
\usepackage[utf8]{inputenc}

\title{INFORMACIONI SISTEMI\\Kovid sistem}
\author{
Bogdan Bojović 1019/2021\\
Kosta Grujčić 1012/2021\\
Miodrag Radojević 1012/2020\\
Luka Đorović 1029/2021\\
Irena Vasiljević 1018/2021
}

\date{Beograd, 2021.}

\begin{document}

\maketitle
\tableofcontents

\newpage

\section{Uvod}

\section{Analiza sistema}

\section{Procesi i slučajevi upotrebe}

%Menjajte strukturu i nazive kako vam se uklapa u slučaj :)

\subsection{Vakcinacija}
\subsubsection{Slučaj upotrebe: Registracija korisnika}
\subsubsection{Slučaj upotrebe: Zahtev za primanje prve doze vakcine}
\subsubsection{Slučaj upotrebe: Obaveštavanje o drugoj/trećoj dozi vakcine}

\subsection{Informisanje}
\subsubsection{Slučaj upotrebe: Sprovođenje statistike}
\subsubsection{Slučaj upotrebe: Kol-centar}

\subsection{Kovid propusnice}

U ovom poglavlju se bavimo formalizacijom rada sa kovid propusnicama. U nastavku je opisan slučaj upotrebe podnošenja zahteva i izdavanja kovid propusnice.

\subsubsection{Slučaj upotrebe: Izdavanje kovid propusnice}

%Prvih 5 tacaka je obavezno, ostale nisu - navesti potrebne
%Ne bi bilo loše da, ako se dodaje neki dijagram, stoji opis uz njega i da se negde u tekstu onda reveriše na tu sliku

\begin{itemize}
    \item \textbf{Kratak opis:} Korisnik bira opciju za podnošenje zahteva za kovid propusnicu. Sistem proverava podatke, izdaje propusnicu i vraća odgovarajuću poruku.
    \item \textbf{Učesnici:}
        \begin{itemize}
            \item Korisnik - želi brzo da dobije kovid propusnicu uz minimalan broj koraka
        \end{itemize}
    \item \textbf{Preduslovi:} Sistem je aktivan. Korisnik je registrovan i ima pristup internetu.
    \item \textbf{Postuslovi:} Korisnik je dobio kovid propusnicu.
    \item \textbf{Osnovni tok:}
        \begin{enumerate}
            \item Korisnik otvara stranicu za prijavu.
            \item Sistem prikazuje formular za prijavu.
            \item Korisnik unosi odgovarajuće podatke.
            \item Korisnik potvrđuje unos.
            \item Sistem vrši validaciju podataka.
            \item Korisnik otvara stranicu za podnošenje zahteva za izdavanje propusnice.
            \item Sistem prikazuje tri moguće opcije:
                \begin{itemize}
                    \item Izdavanje propusnice na osnovu primljene vakcine.
                    \item Izdavanje propusnice na osnovu negativnog PCR ili antigenskog testa.
                    \item Izdavanje propusnice na osnovu preležanog virusa.
                \end{itemize}
            \item Korisnik bira jednu od ponuđenih opcija.
            \item Sistem proverava zadovoljenost uslova.
            \item Sistem korisniku šalje mejl sa kovid propusnicom.
            \item Sistem obaveštava korisnika da je operaciju uspešno izvršena i da je propusnica poslata.
        \end{enumerate}
     
    %Ako postoje, podtokove nazivati sa P1, P2, ...   
    %\item \textbf{Podtokovi:}    
    
    %Alternativne tokove nazivati sa A1, A2, ...
    \item \textbf{Alternativni tokovi:}
        \begin{itemize}
            \item[A1.] \textbf{Neuspešno prijavljivanje.} Ukoliko u koraku 5 osnovnog toka sistem naiđe na neispravne podatke, obaveštava korisnika i zahteva ponovni unos podataka. Proces se nastavlja u koraku 3 osnovnog toka.
            \item[A2.] \textbf{Uslovi za izdavanje nisu zadovoljeni.} Ukoliko u koraku 9 osnovnog toga sistem ustanovi da nisu zadovoljeni odgovarajući uslovi za izdavanje propusnice, obaveštava korisnika. Proces se završava.
        \end{itemize}
    
    %Ako postoje, specijalne zahteve navesti ovde
    %\item \textbf{Specijalni zahtevi:}
        
    \item \textbf{Dodatne informacije:}
        \begin{itemize}
            \item Podaci koji su potrebni za prijavu na sistem su korisničko ime i lozinka.
            \item Uslovi za izdavanje propusnice su:
                \begin{itemize}
                    \item Druga doza vakcine je primljena pre manje od 7 meseci ili  je primljena treća doza vakcine.
                    \item Postojanje negativnog PCR testa koji nije stariji od 72 sata ili antigenskog testa koji nije stariji od 48 sati.
                    \item Virus je preležan pre manje od 7 meseci.
                \end{itemize}
            \item Kovid propusnica sadrži QR kod na osnovu kog se proverava validnost propusnice.
        \end{itemize}
\end{itemize}

\subsubsection{Slučaj upotrebe: Validacija kovid propusnica}

\subsection{Testiranje}

\subsubsection{Slučaj upotrebe: Unos informacija o urađenom testu}

\end{document}
